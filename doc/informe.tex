% Distintas opciones de impresion:
% - Para la entrega final:
\documentclass[a4paper, onecolumn, 12pt, final]{article}

% - Por si hay que ir imprimiendo versiones:
%\documentclass[a4paper, onecolumn, 11pt, draft]{article}

\usepackage[utf8]{inputenc}
% Si eliminamos esta línea la tipografía queda como la original (creo).
\usepackage[T1]{fontenc}

% Paquete donde especificamos el idioma del documento.
\usepackage[catalan]{babel}

\usepackage[pdftex]{color, graphicx}
% Este paquete establece el formato del titulo de los caption (por ejemplo para
% ponerlo en una tabla o figura).
\usepackage[font=small, format=plain, labelfont=bf, up, textfont=it, up]{caption}
\usepackage{url}
\usepackage[plainpages=false, pdfpagelabels, bookmarks=true]{hyperref}
\usepackage{textcomp}
\usepackage{sverb}
\usepackage{verbatim}
\usepackage[kerning, spacing]{microtype}
\microtypecontext{spacing=nonfrench}
\usepackage{url}
\usepackage{hyperref}
\usepackage{makeidx}
\usepackage{amssymb, amsmath}
% No encuentra el paquete
%\usepackage{lgrind}

% Para resaltar la sintaxis del código fuente:
% - Definimos una serie de colores
\usepackage{color}
\definecolor{gray45}{gray}{.45}
\definecolor{gray97}{gray}{.97}
\usepackage{listings}
\lstset{
  language=Ada,
  basicstyle=\small\ttfamily,
  numbers=left,
  numberstyle=\tiny,
  stepnumber=1,
  numbersep=5pt,
  backgroundcolor=\color{gray97},
  showspaces=false,
  showtabs=false,
  tabsize=4,
  breakatwhitespace=false,
  showstringspaces=false,
  keywordstyle=\color{blue},
}

% - Minimizar fragmentado de listados
%\lstnewenvironment{listing}[1][]
%{\lstset{#1}\pagebreak[0]}{\pagebreak[0]}

%\lstdefinestyle{terminal}
%{basicstyle=\scriptsize\bf\ttfamily,
%backgroundcolor=\color{gray75},
%}

% - Definimos el lenguaje Ada
%\lstdefinestyle{Ada}{language=Ada}


\pagestyle{headings}
\usepackage[format=plain,labelfont=bf,up,textfont=it,up]{caption}
\pagestyle{headings}

\newpage
\makeindex
\author{
  José Ruiz Bravo, \small{123456789 <joseruizbravo@gmail.com>},\\
  Biel Moyà Alcover, \small{43142617E <bilibiel@gmail.com>},\\
  Álvaro Medina Ballester, \small{43176576X <alvaro@comiendolimones.com>}
}

\date{data d'entrega}

\begin{document}
\begin{titlepage}

\begin{center}


% Upper part of the page
\includegraphics[width=0.15\textwidth]{logo}\\[1cm]    

\textsc{\LARGE Universitat de les Illes Balears}\\[1.5cm]

\textsc{\Large Processadors del Llenguatge}\\[0.5cm]


% Title
\HRule \\[0.4cm]
{ \huge \bfseries Pràctica de Compiladors. El compilador ``Compilemon''.}\\[0.4cm]

\HRule \\[1.5cm]

% Author and supervisor
\begin{minipage}{0.4\textwidth}
\begin{flushleft} \large
\emph{Autors:}\\
\small José \textsc{Ruiz Bravo} <joseruizbravo@gmail.com>\\
\small Biel \textsc{Moyà Alcover}  <bilibiel@gmail.com>\\
\small Álvaro \textsc{Medina Ballester}  <alvaro@comiendolimones.com>\\
\end{flushleft}
\end{minipage}
\begin{minipage}{0.4\textwidth}
\begin{flushright} \large
\emph{Professor:} \\
Dr.~Albert \textsc{Llemosí Cases}
\end{flushright}
\end{minipage}

\vfill

% Bottom of the page
{\large \today}

\end{center}

\end{titlepage}


\tableofcontents
\newpage

\begin{abstract}
Compilador \emph{compilemon} creat amb el llenguatge Ada. Està
composat per un subconjunt bàsic d'instruccions en Ada conegudes
com \emph{lemonada}.
\end{abstract}

\thispagestyle{empty}

\section{Introducció}
Al present document trobam el codi font del compilador \emph{Compilemon},
desenvolupat per l'assignatura Processadors del Llenguatge, impartida per el
Doctor Albert Llemosí Cases. Per desenvolupar la pràctica hem fet servir el
llenguatge Ada 2005, juntament amb les eines següents:

\begin{description}
\item [Edició de text] Com a editor de text hem fer servir GNU Emacs i Gedit com
a editors principals, tant per el codi de la pràctica com per el de l'informe.
\item [Control de versions] Per duur a terme el desenvolupament de forma
conjunta, hem emprat el sistema de control de versions Subversion. Podem trobar
el repositori a la direcció \url{http://svn.comiendolimones.com/compilemon}.
\item [Documentació] Per elaborar la documentació hem fet servir \LaTeX~com a
editor de text.
\item [Sistema Operatiu] Tota la pràctica s'ha desenvolupat sota entorns
UNIX. Primerament sobre una màquina Darwin BSD i més endavant sota GNU/Linux.
\end{description}
\newpage

\section{Anàlisi Lèxica}

\subsection{Descripció del lèxic: \emph{pk_ulexica.l}}
\lstinputlisting[style=LEM]{../src/pk_ulexica.l}
\newpage

\subsection{Taula de noms}
\subsubsection{Fitxer \emph{decls-d\_taula\_de\_noms.ads}}
\lstinputlisting{../src/decls-d_taula_de_noms.ads}
\newpage

\subsubsection{Fitxer \emph{decls-d\_taula\_de\_noms.adb}}
\lstinputlisting{../src/decls-d_taula_de_noms.adb}
\newpage

\subsection{Tokens i atributs}
\subsubsection{Fitxer \emph{d\_token.ads}}
\lstinputlisting{../src/d_token.ads}
\newpage

\subsubsection{Fitxer \emph{decls-d\_atribut.ads}}
\lstinputlisting{../src/decls-d_atribut.ads}
\newpage

\subsubsection{Fitxer \emph{decls-d\_atribut.adb}}
\lstinputlisting{../src/decls-d_atribut.adb}
\newpage

\newpage
\section{Anàlisi Sintàctica}

\subsection{Gramàtica del nostre llenguatge}
% Gramàtica del nostre llenguatge

    \begin{tabbing}
    \hspace*{2.0cm} \= \hspace*{0.5cm} \= \hspace*{0.8cm} \= \kill
        \textit{programa} \> $\rightarrow $ \> \textit{procediment} \\
        \\
        \textit{procediment} \> $ \rightarrow $ \> \textbf{PC\_PROCEDURE} \textit{encap} \textbf{PC\_IS} \\
        \> \> \> \textit{declaracions} \\
        \> \> \textbf{PC\_BEGIN} \\
        \> \> \> \textit{bloc} \\
        \> \> \textbf{PC\_END} identificador; \\
        \\
        \textit{encap} \> $ \rightarrow $ \> identificador \textit{args} \\
        \\
        \textit{args} \> $\rightarrow$ \> (\textit{lparam}) \\
        \> $ \mid $ \> $ \lambda $ \\
        \\
        \textit{lparam} \> $\rightarrow$ \> \textit{lparam} ; \textit{param} \\
        \> $\mid$ \> \textit{param} \\
        \\
        \textit{param} \> $\rightarrow$ \> identificador : \textit{mode} identificador \\
        \\
        \textit{mode} \> $\rightarrow$ \> \textbf{PC\_IN} \\
        \> $\mid$ \> \textbf{PC\_OUT} \\
        \> $\mid$ \> \textbf{PC\_IN PC\_OUT} \\
        \\
        \textit{declaracions} \> $\rightarrow$ \> \textit{declaracions} \textit{declaracio} \\
        \> $\mid$ \> $\lambda$ \\
        \\
        \textit{declaracio} \> $\rightarrow$ \> \textit{dec\_var} \\
        \> $\mid$ \> \textit{dec\_constant} \\
        \> $\mid$ \> \textit{dec\_tipus} \\
        \> $\mid$ \> \textit{programa} \\
        \\
        
        \underline{-- Manual d'usuari variables} \\
        \textit{dec\_var} \> $\rightarrow$ \> \textit{lid} : identificador; \\
        \\
        \textit{lid} \> $\rightarrow$ \> \textit{lid}, identificador \\
        \> $\mid$ \> identificador \\
        \\
        
        \underline{-- Manual d'usuari constant} \\
        \textit{dec\_constant} \> $\rightarrow$ \> identificador : \textbf{PC\_CONSTANT} identificador := \textit{valor};  \\
        \\
        \textit{valor} \> $\rightarrow$ \> \textit{lit} \\
        \> $\mid$ \> $-$ \textit{lit} \\
        \\
        %\textit{lit} \> $\rightarrow$ \> lit\_num \\
        %\> $\mid$ \> lit\_car \\
        %\> $\mid$ \> lit\_string \\
        %\\
        
        \underline{-- Manual d'usuari tipus} \\
        \textit{dec\_tipus} \> $\rightarrow$ \> \textit{dec\_subrang}  \\
        \> $\mid$ \> \textit{dec\_registre} \\
        \> $\mid$ \> \textit{dec\_coleccio} \\
        \\
        \textit{dec\_subrang} \> $\rightarrow$ \> \textbf{PC\_TYPE} identificador \textbf{PC\_IS PC\_NEW} identificador 
			\\\> \textbf{PC\_RANGE} \textit{valor} .. \textit{valor}; \\
        \\
        \textit{dec\_registre} \> $\rightarrow$ \> \textbf{PC\_TYPE} identificador \textbf{PC\_IS PC\_RECORD} \\
        \> \> \> \textit{ldc} \\
        \> \> \textbf{PC\_END PC\_RECORD}; \\
        \\
        \textit{ldc} \> $\rightarrow$ \> \textit{ldc dc} \\
        \> $\mid$ \> \textit{dc} \\
        \\
        \textit{dc} \> $\rightarrow$ \> identificador : identificador; \\
        \\
        
        \underline{-- Tipus colecció (\textit{array})} \\
        \textit{dec\_coleccio} \> $\rightarrow$ \> \textbf{PC\_TYPE} identificador \textbf{PC\_IS PC\_ARRAY} \\
		\> (\textit{lid}) \textbf{PC\_OF} identificador; \\
        \\
        \textit{lid} \> $\rightarrow$ \> \textit{lid}, identificador \\
        \> $\mid$ \> identificador \\
        \\
        
        
        \underline{-- Bloc d'instruccions} \\
        \textit{bloc} \> $\rightarrow$ \> \textit{bloc sent} \\
        \> $\mid$ \> \textit{sent} \\
        \\
        \textit{sent} \> $\rightarrow$ \> \textit{sassig} \\
        \> $\mid$ \> \textit{scond} \\
        \> $\mid$ \> \textit{srep} \\
        \> $\mid$ \> \textit{crida\_proc} \\
		\> $\mid$ \> $\lambda$ \\
        \\
        
        \textit{sassig} \> $\rightarrow$ \> \textit{referencia} := \textit{expressio}; \\
        \\
        
        \textit{scond} \> $\rightarrow$ \> \textbf{PC\_IF} \textit{expressio} \textbf{PC\_THEN} \\
        \> \> \>  \textit{bloc} \\
        \> \> \textbf{PC\_END PC\_IF}; \\
        \> $\mid$ \> \textbf{PC\_IF} \textit{expressio} \textbf{PC\_THEN} \\
        \> \> \> \textit{bloc} \\
        \> \> \textbf{PC\_ELSE} \\
        \> \> \> \textit{bloc} \\
        \> \> \textbf{PC\_END PC\_IF}; \\
        \\
        
        \textit{srep} \> $\rightarrow$ \> \textbf{PC\_WHILE} \textit{expressio} \textbf{PC\_LOOP} \\
        \> \> \> \textit{bloc} \\
        \> \> \textbf{PC\_END PC\_WHILE}; \\
        \\
        
        \textit{crida\_proc} \> $\rightarrow$ \> \textit{referencia}; \\
        \\
        \textit{referencia} \> $\rightarrow$ \> identificador \\
        \> $\mid$ \> \textit{referencia}.identificador \\
        \> $\mid$ \> \textit{referencia} (\textit{prparam}) \\
        \\
        \textit{prparam} \> $\rightarrow$ \> \textit{expressio} \\
        \> $\mid$ \> \textit{expressio, prparam} \\
        \\
        
        \textit{expressio} \> $\rightarrow$ \> \textit{expressio} $ + $ \textit{expressio} \\
        \> $\mid$ \> \textit{expressio} $ - $ \textit{expressio} \\
        \> $\mid$ \> \textit{expressio} $ * $ \textit{expressio} \\
        \> $\mid$ \> \textit{expressio} $ / $ \textit{expressio} \\
        \> $\mid$ \> \textit{expressio} \textbf{PC\_MOD} \textit{expressio} \\
        \> $\mid$ \> \textit{expressio} $ > $ \textit{expressio} \\
        \> $\mid$ \> \textit{expressio} $ < $ \textit{expressio} \\
        \> $\mid$ \> \textit{expressio} $ \geq $ \textit{expressio} \\
        \> $\mid$ \> \textit{expressio} $ \leq $ \textit{expressio} \\
        \> $\mid$ \> \textit{expressio} $ \neq $ \textit{expressio} \\
        \> $\mid$ \> \textit{expressio} $ = $ \textit{expressio} \\
        \> $\mid$ \> $-$ \textit{expressio} \\
        \> $\mid$ \> \textit{expressio} \&\& \textit{expressio} \\
        \> $\mid$ \> \textit{expressio} $\mid\mid$ \textit{expressio} \\
        \> $\mid$ \> \textbf{PC\_NOT} \textit{expressio} \\
        \> $\mid$ \> (\textit{expressio}) \\
        \> $\mid$ \> \textit{referencia} \\
        \> $\mid$ \> \textit{lit} \\
    \end{tabbing}

\newpage

%\subsection{Especificació \emph{pk\_usintactica.y}}
%\lstinputlisting{../src/pk_usintactica.y}
%\newpage

\newpage

%Per la segona entrega:
\section{Anàlisi Semàntica}

\subsection{Taula de simbols}
\subsubsection{Fitxer \emph{decls-dtsimbols.ads}}
\lstinputlisting{../src/decls-dtsimbols.ads}
\newpage

\subsubsection{Fitxer \emph{decls-dtsimbols.adb}}
\lstinputlisting{../src/decls-dtsimbols.adb}
\newpage

\subsection{Descripció}
\subsubsection{Fitxer \emph{decls-dtdesc.ads}}
\lstinputlisting{../src/decls-dtdesc.ads}
\newpage

\subsection{Semàntica}
\subsubsection{Fitxer \emph{semantica.ads}}
\lstinputlisting{../src/semantica.ads}
\newpage

\subsubsection{Fitxer \emph{semantica.adb}}
\lstinputlisting{../src/semantica.adb}
\newpage

\subsection{Comprovació de tipus}
\subsubsection{Fitxer \emph{decls-dtnode.ads}}
\lstinputlisting{../src/decls-dtnode.ads}
\newpage

%% \subsubsection{Fitxer \emph{decls-d\_arbre.ads}}
%% \lstinputlisting{../src/decls-d_arbre.ads}
%% \newpage

\subsubsection{Fitxer \emph{decls-d\_arbre.adb}}
\lstinputlisting{../src/decls-d_arbre.adb}
\newpage

\subsubsection{Fitxer \emph{semantica-ctipus.ads}}
\lstinputlisting{../src/semantica-ctipus.ads}
\newpage

\subsubsection{Fitxer \emph{semantica-ctipus.adb}}
\lstinputlisting{../src/semantica-ctipus.adb}
\newpage

\subsection{Missatges d'error}
\subsubsection{Fitxer \emph{semantica-missatges.ads}}
\lstinputlisting{../src/semantica-missatges.ads}
\newpage

\subsubsection{Fitxer \emph{semantica-missatges.adb}}
\lstinputlisting{../src/semantica-missatges.adb}
\newpage

\newpage

%Tercera entrega i final
\section{Generació de codi intermedi}
\subsection{Codi de 3 adreces}
\subsubsection{Fitxer \emph{semantica-declsc3a.ads}}
\lstinputlisting{../src/semantica-declsc3a.ads}
\newpage

\subsubsection{Fitxer \emph{semantica-declsc3a.adb}}
\lstinputlisting{../src/semantica-declsc3a.adb}
\newpage

\subsection{Piles}
\subsubsection{Fitxer \emph{piles.ads}}
\lstinputlisting{../src/pilas.ads}
\newpage

\subsubsection{Fitxer \emph{piles.adb}}
\lstinputlisting{../src/pilas.adb}
\newpage

\subsection{Generació de codi intermedi}
\subsubsection{Fitxer \emph{semantica-gci.ads}}
\lstinputlisting{../src/semantica-gci.ads}
\newpage

\subsection{Generació de codi intermedi}
\subsubsection{Fitxer \emph{semantica-gci.adb}}
\lstinputlisting{../src/semantica-gci.adb}
\newpage

\newpage
\section{Assemblador}
\subsection{Generació de codi assemblador}
\subsubsection{Fitxer \emph{semantica-assemblador.ads}}
\lstinputlisting{../src/semantica-assemblador.ads}
\newpage

\subsubsection{Fitxer \emph{semantica-assemblador.adb}}
\lstinputlisting{../src/semantica-assemblador.adb}
\newpage

\newpage
\section{Proves i programa principal}

\subsection{Fitxer \emph{compilemon.adb}, programa principal}
\lstinputlisting{../src/compilemon.adb}
\newpage

\newpage

\section{Declaracions i altres paquets}

\subsection{Fitxer \emph{decls.ads}}
\lstinputlisting{../src/decls.ads}
\newpage

\subsection{Fitxer \emph{decls-dgenerals.ads}}
\lstinputlisting{../src/decls-dgenerals.ads}
\newpage

%\subsection{Fitxer \emph{decls-d\_hash.ads}}
%\lstinputlisting{../src/decls-d_hash.ads}
%\newpage

%\subsection{Fitxer \emph{decls-d\_hash.adb}}
%\lstinputlisting{../src/decls-d_hash.adb}
%\newpage

\newpage

\section{Jocs de proves}
\subsection{Arrays de records}
\subsubsection{Fitxer \emph{prova1.lem}}
\lstinputlisting{../src/prova1.lem}
\newpage
\subsubsection{Fitxer \emph{prova1.lem.c3at}}
\lstinputlisting{../src/prova1.lem.c3at}
\newpage

\subsection{Suma de matrius 3x3}
\subsubsection{Fitxer \emph{prova2.lem}}
\lstinputlisting{../src/prova2.lem}
\newpage
\subsubsection{Fitxer \emph{prova2.lem.c3at}}
\lstinputlisting{../src/prova2.lem.c3at}
\newpage

\subsection{Vector amb rangs negatius}
\subsubsection{Fitxer \emph{prova3.lem}}
\lstinputlisting{../src/prova3.lem}
\newpage
\subsubsection{Fitxer \emph{prova3.lem.c3at}}
\lstinputlisting{../src/prova3.lem.c3at}
\newpage
\subsubsection{Fitxer \emph{prova3.lem.s}}
\lstinputlisting[language={[x86masm]Assembler}]{../src/prova3.lem.s}
\newpage

\subsection{Algoritme d'ordenació \emph{Quicksort}}
\subsubsection{Fitxer \emph{prova4.lem}}
\lstinputlisting{../src/prova4.lem}
\newpage
\subsubsection{Fitxer \emph{prova4.lem.c3at}}
\lstinputlisting{../src/prova4.lem.c3at}
\newpage

\subsection{Multiplicació e inversió de matrius 5x5}
\subsubsection{Fitxer \emph{prova5.lem}}
\lstinputlisting{../src/prova5.lem}
\newpage
\subsubsection{Fitxer \emph{prova5.lem.c3at}}
\lstinputlisting{../src/prova5.lem.c3at}
\newpage

\newpage

\appendix
\section{Conclusions}
S'han desenvolupat 6 207 linies de codi Ada i 371 linies de codi yacc. Segons
el model COCOMO, l'esforç dels desenvolupadors és de 1,48 persones-anys. El
nombre estimat de desenvolupadors és de 2,38 i el cost de desenvolupar el
treball és de 199,326 \$.

El desenvolupament d'una pràctica gran i complexa com aquesta ens ha plantejat
un conjunt de reptes. En altres pràctiques, no havíem necessitat un control tan
gran sobre els fitxers generats i sobre les versions de codi generades. El fet
d'esser 3 persones programant ha fet que ens haguem hagut de dividir les feines i
organitzar de tal forma que no hi hagi hagut colissions i que tots els bossins
hagin hagut d'encaixar de forma perfecta. Personalment, trobam que ha estat un
repte haver desenvolupat el nostre compilador, i hem aprés molts conceptes
mentres ho feiem.


\end{document}

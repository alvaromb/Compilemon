% Distintas opciones de impresion:
% - Para la entrega final:
\documentclass[a4paper, onecolumn, 12pt, final]{article}

% - Por si hay que ir imprimiendo versiones:
%\documentclass[a4paper, onecolumn, 10pt, final]{article}

\usepackage[utf8]{inputenc}
% Si eliminamos esta línea la tipografía queda como la original (creo).
\usepackage[T1]{fontenc}

% Paquete donde especificamos el idioma del documento.
\usepackage[catalan]{babel}

\usepackage[pdftex]{color, graphicx}
% Este paquete establece el formato del titulo de los caption (por ejemplo para
% ponerlo en una tabla o figura).
\usepackage[font=small, format=plain, labelfont=bf, up, textfont=it, up]{caption}
\usepackage{url}
\usepackage[plainpages=false, pdfpagelabels, bookmarks=true]{hyperref}
\usepackage{textcomp}
\usepackage{sverb}
\usepackage{verbatim}
\usepackage[kerning, spacing]{microtype}
\microtypecontext{spacing=nonfrench}
\usepackage{url}
\usepackage{hyperref}
\usepackage{makeidx}
\usepackage{amssymb, amsmath}
% No encuentra el paquete
%\usepackage{lgrind}

% Para resaltar la sintaxis del código fuente:
% - Definimos una serie de colores
\usepackage{color}
\definecolor{gray45}{gray}{.45}
\definecolor{gray97}{gray}{.97}
\usepackage{listings}
\lstset{
  language=Ada,
  basicstyle=\small\ttfamily,
  numbers=left,
  numberstyle=\tiny,
  stepnumber=1,
  numbersep=5pt,
  backgroundcolor=\color{gray97},
  showspaces=false,
  showtabs=false,
  tabsize=8,
  breakatwhitespace=false,
  showstringspaces=false,
  keywordstyle=\scshape\color{blue},
  commentstyle=\color{gray45},
  stringstyle=\color{red},
}

% - Minimizar fragmentado de listados
%\lstnewenvironment{listing}[1][]
%{\lstset{#1}\pagebreak[0]}{\pagebreak[0]}

%\lstdefinestyle{terminal}
%{basicstyle=\scriptsize\bf\ttfamily,
%backgroundcolor=\color{gray75},
%}

% - Definimos el lenguaje Ada
\lstdefinestyle{LEM}{
basicstyle=\small\bf\ttfamily,
stringstyle=\color{black},
keywordstyle=\bf\color{blue}
}


\pagestyle{headings}
\usepackage[format=plain,labelfont=bf,up,textfont=it,up]{caption}
\pagestyle{headings}

\newpage
\makeindex
\author{
  José Ruiz Bravo, \small{123456789 <joseruizbravo@gmail.com>},\\
  Biel Moyà Alcover, \small{43142617E <bilibiel@gmail.com>},\\
  Álvaro Medina Ballester, \small{43176576X <alvaro@comiendolimones.com>}
}

\date{20 de novembre de 2009}

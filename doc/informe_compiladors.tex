\documentclass[10pt]{article}
\usepackage{ifpdf}
\usepackage[utf8]{inputenc}
\usepackage[catalan]{babel}
\usepackage{makeidx}
\usepackage{lgrind}
\usepackage{color}
\definecolor{gray97}{gray}{.97}
\definecolor{gray75}{gray}{.75}
\definecolor{gray45}{gray}{.45}

\usepackage{listings}
\lstset{ frame=Ltb,
     framerule=0pt,
     aboveskip=0.5cm,
     framextopmargin=3pt,
     framexbottommargin=3pt,
     framexleftmargin=0.4cm,
     framesep=0pt,
     rulesep=.4pt,
     backgroundcolor=\color{gray97},
     rulesepcolor=\color{black},
     stringstyle=\ttfamily,
     showstringspaces = false,
     basicstyle=\small\ttfamily,
     commentstyle=\color{gray45},
     keywordstyle=\color{blue},
     numbers=left,
     numbersep=15pt,
     numberstyle=\tiny,
     numberfirstline = false,
     breaklines=true,
   }
 
% minimizar fragmentado de listados
\lstnewenvironment{listing}[1][]
   {\lstset{#1}\pagebreak[0]}{\pagebreak[0]}
 
\lstdefinestyle{consola}
   {basicstyle=\scriptsize\bf\ttfamily,
    backgroundcolor=\color{gray75},
   }
 
\lstdefinestyle{Ada}
   {language=Ada,
   }

\title{Compiladors, primera entrega}
\author{José Ruiz Bravo, Biel Moyà Alcover, Álvaro Medina Ballester}
\date{21/11/08}
\ifpdf
\pdfinfo {
	/Author (José Ruiz Bravo, Biel Moyà Alcover, Álvaro Medina Ballester)
	/Title (Compiladors, primera entrega)
	/Subject (Compiladors, primera entrega)
	/Keywords (compiladors, primera entrega, gramàtica, léxic)
	/CreationDate (D:20081120195650)
}
\fi
\makeindex
\begin{document}
	\maketitle
	\index{Introducció}
	\section{Introducció}
		La pràctica de l'assignatura \textit{1470 - Processadors del Llenguatge} consisteix en
		desenvolupar un compilador basat en el llenguatge Ada. Nosaltres hem decidit 
		utilitzar un subconjunt d'instruccions d'Ada traduïdes al català, per crear
		un llenguatge simple, senzill i fàcilment entendible.
		\\
		\\
		En aquest document trobarem el codi font necessari per definir i utilitzar la
		taula de noms, el codi font del lèxic del llenguatge i l'especificació de la
		gramàtica del nostre llenguatge. Aquests documents es corresponen amb les primeres
		etapes del desenvolupament d'un compilador: l'anàlisi lèxica i l'anàlisi 			sintàctica\footnote{L'anàlisi sintàctica es completarà en posteriors versions de la 			pràctica.}.
	\newpage
	\section{Anàlisi Lèxica}
		El nostre llenguatge el definirem a partir d'una gramàtica incontextual. Mitjançant l'eina \textit{aflex}, generam el codi necessari per representar el lèxic que hem definit. Els \textit{tokens} que composen l'apartat lèxic del compilador son els següents (fitxer \textit{compilemon.l}):
		\\
	
	% Codigo del análisis léxico
	\lgrindfile{compilemon.l.tex}
	
	\newpage
	\section{Estructura de dades}
		Hem definit una estructura de dades per emmagatzemar els identificadors i el strings. Aquesta estructura de dades es coneix com a \textit{taula de noms} i es similar a la que vàrem definir a classe.
		\\
		
	% Código de decls-d_taula_de_noms.ads
	\begin{lstlisting}[style=Ada]
-- ------------------------------------------------
--  Paquet de declaracions de la taula de noms
-- ------------------------------------------------
--  Versio	:	0.1
--  Autors	:	Jose Ruiz Bravo
--			Biel Moya Alcover
--			Alvaro Medina Ballester
-- ------------------------------------------------
--	Especificacio de l'estructura necessaria
-- per el maneig de la taula de noms i dels metodes
-- per tractar-la.
--
-- ------------------------------------------------

with 	decls.dgenerals,
	decls.d_hash; 
	
use 	decls.dgenerals,
	decls.d_hash;


package decls.d_taula_de_noms is

	pragma pure;

	-- Excepcions
	E_Tids_Plena : exception;
	E_Tcar_Plena : exception;

	type taula_de_noms is limited private;

	procedure tbuida	(tn : out taula_de_noms);

	procedure posa_id 	(tn : in out taula_de_noms;  
				idn : out id_nom; 
				nom : in string);

	procedure posa_str	(tn : in out taula_de_noms;
				idn : out id_nom;
				  s : in string );
					
	function cons		(tn : in taula_de_noms; 
				idn : in id_nom) return string;
							

	private
	
		longitut : constant integer := 40;
	
		-- La longitud es el nombre de paraules * la longitud de cadascuna
		type rang_tcar is new integer range 0 .. (longitut*max_id)-1;
	
		type t_identificador is record 
			pos_tcar : rang_tcar;
			 seguent : id_nom;	  	
                    long_paraula : Natural;
		end record;
	
		type taula_identificadors is array (id_nom) of t_identificador;
	
		type taula_caracters is array (rang_tcar) of character; 
	
		type taula_de_noms is record
			  td : taula_dispersio;
			 tid : taula_identificadors;
			  tc : taula_caracters;
			 nid : id_nom;
			ncar : rang_tcar;
		end record;
	
	
		-- Funcio de comparacio de dues paraules
		function par_iguals (par1, par2 : in string) return boolean;		
				
		
end decls.d_taula_de_noms;		
	\end{lstlisting}
		
		
\end{document}

